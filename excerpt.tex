\section*{Excerpts}

\subsection*{Quality and Productivity Outcomes Relating to Continuous Integration in GitHub}

\textcite[]{Vasilescu2015} defined Software process integration and automation as areas of key concern in software engineering. In their work they want to analyse if these innovations actually help projects. 

Given the numerous confounding factors that can influence project performance, it can be a challenge to discern the effects of process integration and automation. \autocite[p. 2]{Vasilescu2015}

Their main finding is that continuous integration improves the productivity of project teams, who can integrate more outside contributions, without an observable diminishment in code quality. \autocite[]{Vasilescu2015}

The findings show the benefits of introducing Continuous Integration to pull request process: ``more pull requests get processed; more are being accepted and merged, and more are also being rejected. Moreover, this increased productivity doesn't appear to be gained at the expense of quality.''\autocite[]{Vasilescu2015}

How important is the paper for my research field?

\subsection*{A Hundred Days of Continuous Integration}

\textcite[]{Miller2008} report outlines their team experience with Continuous Integration and try to answer the following questions:
\begin{quote}
``Does the effort of maintaining the CI server and fixing build breaks save time compared to a lengthier check-in process that attempts to never break the build?'' \autocite[]{Miller2008}
\end{quote}
\begin{quote}
``How do you convince teams and management that it's worth adopting and how best to do it?'' \autocite[]{Miller2008}
\end{quote}
His research paper provides a list of recommendations and findings:

\noindent Build Process
\begin{itemize}
    \item Treat warnings as errors.
    \item If you break the build, you fix it.
    \item Don't check in and go home.
\end{itemize}
Team Organization
\begin{itemize}
    \item Co-locate by feature not discipline.
    \item pay the taxes associated with distribution
    \item Focus on team consistency
\end{itemize}

\noindent In his work \textcite[]{Miller2008} analysis the data and experience of their distributed Team he write about challenges and benefits from conducting CI.

\subsection*{Experienced Benefits of Continuous Integration in Industry Software Product Development: A Case Study}

\textcite[]{Stahl2013} presents a multi case study of industrial experiences of continuous integration among software professionals working on large scale development projects.

They try to validate the benefits of Continuous Integration in following aspects: ``increased developer productivity, increased project predictability, improved communication and enabling agile testing.''\autocite[]{Stahl2013}

The study involves four independent products and in each of these products they interviewed developers, testers, project managers and line managers. Their experiences of Continuous integration are discussed in comparison to the benefits proposed in related work.
\\
\\
\noindent The paper addresses 4 hypotheses:
\begin{enumerate}
    \item CI supports the agile testing practices.
    \item CI contributes to improved communication.
    \item CI contributes to increased developer productivity.
    \item CI improves project predictability as an effort of finding problems earlier.
\end{enumerate}

\noindent \textcite[]{Stahl2013} concluded that CI has not only one, but several benefits and each of their four hypotheses are at least partly supported by the gathered data.